\documentclass[12pt,oneside]{amsart}
\title{Math 522 Syllabus}

\usepackage[vscale=.8]{geometry}
\usepackage{setspace}\onehalfspacing
\renewcommand{\labelitemi}{$\circ$}

\begin{document}
\maketitle
\thispagestyle{empty}

\subsection*{Course information}
\begin{description}
\item[Meeting times] W,F from 9:00 to 10:15am
\item[Meeting place] M 124
\item[Text] Kunen, \emph{Set Theory} (Studies in Logic edition)
\item[My email] \texttt{scoskey@boisestate}
\item[My office] MG 237-A
\item[Office hours] W 12pm, Th 1pm, and by appointment
\end{description}

\subsection*{Course topics}

This course is about sets of real numbers, and more specifically about mesauring the \emph{size} of sets of real numbers. But the word ``size'' means different things to different mathematicians: three important examples are cardinality, measure, and category. \emph{Cardinality} asks whether a set is countable or uncountable; \emph{measure} asks whether a set has zero length or nonzero length, and \emph{category} asks whether a set is meager or nonmeager. We will ask: how do these three kinds of size compare with one another?

Our investigation of mesaure and category will lead us to a number of \emph{independence results}, that is, statements that cannot be decided using the usual axioms of set theory. The most central example of such a statement is the \emph{continuum hypothesis}, which asserts that the set of all real numbers has cardinality equal to the first uncountable cardinal number. In order to prove this statement is indeed independent of set theory, we will study a technique called set-theoretic \emph{forcing}.

Once we have developed the machinery of forcing, we will return to the notions of measure and category. We will find that there are numerous cardinal numbers surrounding the zero length sets and the meager sets whose values, like the size of the set of all real numbers, are also independent of set theory.

\subsection*{Homework}
Homework problems will be assigned regularly. To receive a passing grade, you must attempt all problems. To receive an A or B, your solutions should be correct and complete. You may correct and resubmit homework problems within two weeks of receiving them back.

\subsection*{Attendance and participation}
Your attendance and participation are required (a few exceptions are of course allowed). Finally you may be asked to give short presentations about homework solutions or other class material.


\end{document}
